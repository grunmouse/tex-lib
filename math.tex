%частная производная
\newcommand{\pdd}[2]{
	\frac{\partial#1}{\partial#2}
}
\newcommand{\pddII}[3]{
	\frac{\partial^2 #1}{\partial#2\partial#3}
}


%производная по одной переменной
\newcommand{\dd}[2]{
	\frac{d#1}{d#2}
}
%дифференциал
\newcommand{\diff}[1]{
	d#1
}

\newcommand{\abs}[1]{
	\left | #1 \right |
}
\newcommand{\brkt}[1]{
	\left ( #1 \right )
}

%определяем функцию
\newcommand{\deffn}[1]{
	\expandafter\def\csname #1\endcsname{\mspace{2mu}\mathrm{#1}\mspace{1mu}}
}


%Имена функций
\deffn{th}
%\deffn{ch}
%\deffn{sh}

\deffn{acos}
\deffn{asin}

%\deffn{arctg}
\deffn{arth}
\deffn{sign}
%\deffn{tg}
\deffn{ln}


%заплатки
\newcommand{\defgreek}[2]{
	\expandafter\def\csname #1\endcsname{\mathrm{#2}}
}


\defgreek{Alpha}{A}
\defgreek{Beta}{B}
\defgreek{Epsilon}{E}
\defgreek{Zeta}{Z}
\defgreek{Eta}{H}
\defgreek{Iota}{I}
\defgreek{Kappa}{K}
\defgreek{Mu}{M}
\defgreek{Nu}{N}
\defgreek{Omicron}{O}
\defgreek{omicron}{o}	
\defgreek{Rho}{P}
\defgreek{Tau}{T}
\defgreek{Chi}{X}